\documentclass{udparticle}
\setlogo{EIT}
\headertext{Respuestas tarea de preparación para la Solemne}
\title{Tarea de preparación para la Solemne}
\author{Ignacio Yanjari, Dagoberto Navarrete, Ignacio López, Thomas Muñoz.}
\begin{document}
\maketitle
\begin{enumerate}
%aca se iran poniendo las respuestas, para esto escribir \item y su respuesta, eso si estarán en orden.

\item Describa cada uno de los componentes basicos de un sistema de comunicaciones \\
    Fuente:Emite el mensaje\\
    Transductor de entrada: Convierte el mensaje en señales eléctricas\\
    Transmisor: Pasa la señal al canal, mediante el proceso de modulación\\
    Canal de Transmisión: Nexo eléctrico entre el emisor y receptor\\
    Transductor de salida: transforma las señales para que puedan llegar a destino\\
    Receptor: Realiza el proceso de demodulación \\

\item ¿Que es una arquitectura de protocolos? \\
    Estructura en capas de hardware y software que facilita el intercambio de información para que los
    dispositivos se puedan comunicar en forma exitosa\\

\item Resuma las funciones de las siete capas del modelo OSI.\\
    -Aplicación:Conectividad entre red y humano\\
    -Presentación: Representacion común entre los datos transferidos y la capa de aplicacion\\
    -Sesión: Comunicacion entre dispositivos de la red\\
    -Transporte: Conexión extremo-a-extremo y fiabilidad de los datos\\
    -Red:Intercambia los datos individuales en la red de cada dispositivo final, determinacion de ruta e IP\\
    -Enlace de red: Describe los metodos para intercambiar datos, direccionamiento físico MAC y LLC\\
    -Física: Medios mecánico para la transmision y recepción\\
\item Resuma las funciones del modelo TCP/IP híbrido, incluyendo las 2 sub-capas de enlace de datos
(LLC y MAC).\\


\item Mencione algunas ventajas y desventajas del modelo OSI de ISO\\
    -Ventajas:\\
        -Facilita un problema complejo ya que divide este en partes más simples y ordenadas\\
        -Evita los problemas de incompatibilidad\\
        -Los cambios de una capa no afectan las demás capas y éstas pueden evolucionar más rápido\\
    -Desventajas:\\
        -Las capas contienen demasiadas actividades redundantes, por ejemplo, el control de errores se integra en \\
         casi todas las capas siendo que tener un único control en la capa de aplicación o presentación sería suficiente\\
        -La gran cantidad de código que fue necesario para implantar el modelo OSI y su consecuente lentitud hizo que \\
         la palabra OSI fuera interpretada como "calidad pobre" contrastado con el modelo TCP/IP que fue mucho mas efectivo.\\
        -Tecnología desactualizada\\
        
\item Mencione algunas ventajas y desventajas del modelo de referencia TCP/IP híbrido.\\
    NO ENCUENTRO INFORMACION DDD:

\item Indique la diferencia entre un estándar de  facto y un estándar de  jure\\
    El estándar de jure es aprobado por una organización, mientras que el de facto es un estándar aceptado pero 
    no legitimado por una organización\\
\item Defina o explique los conceptos de: interfaz y protocolo\\
    Interfaz: es una conexión entre dos sistemas dando una comunicación en distintos niveles\\
    Protocolo: Conjunto de reglas y estándares que controlan la comunicación entre sistemas\\
\item Explique en que consiste el proceso de encapsulamiento del mensaje\\
    El encapsulamiento rodea los datos con la información de protocolo necesaria antes de que se una al tránsito de la red.
    Por lo tanto, a medida que los datos se desplazan a través de las capas del modelo reciben encabezados, información final y otros
    tipos de información.\\
    Una vez que se envían los datos desde el origen, viajan a través de la capa de aplicación y recorren todas las demás capas en 
    sentido descendente. El empaquetamiento y el flujo de los datos que se intercambian experimentan cambios a medida que las capas 
    realizan sus funciones para los usuarios finales.\\

\itemIndique En que capas del modelo OSI las unidades de datos de los protocolos se llaman segmento,
frame, paquete y bits.\\
    REVISEN ESTO NO ESTOY MUY SEGURO D:
  Física -> Bits\\
  Enlace de red -> Paquete\\
  Red -> Frame\\
  Transporte -> Segmento\\

\item Relacionado con organismos de estandarizacion en redes LAN, ¿qu ´ e significan las siglas ANSI, ´
TIA, EIA, ITU-T, CCITT, FCC, ISO e IEEE?\\

    ANSI: Instituto nacional estadounidense de estándares\\
    TIA: Asociación de la industria de telecomunicaciones\\
    EIA: Alianza de industrias electrónicas\\
    ITU-T: Sector de normalización de las comunicaciones de la UIT\\
    CCITT: Comité Consultivo Internacional Telegráfico y Telefónico\\
    ISO: Organización Internacional de Normalización\\
    IEEE: Instituto de Ingeniería Eléctrica y Electrónica\\

\item Explique lo que entiende por: broadcast (difusión),multicast y unicast.\\
    -broadcast: se envia un paquete a todos los hosts conectados a la red
    -multicast: se envia un paquete a un grupo seleccionado de hosts conectados a la red
    -unicast : envia un paquete a un host determinado de la red
\item Defina y compare los medios simplex, half-duplex y full-duplex. De ejemplos de sistemas para ´
cada uno de estos casos.\\
    simplex: transmite solo en una direccion \\
        emjemplo: cuando se imprime a traves de un computador a una impresora\\
    half-duplex : Transmite en los 2 sentidos pero no a la vez \\
        ejemplo : wokitokis\\
    full-duplex : Transmite en los 2 sentidos simultaneamente\\
        ejemplo : Llamada en telefono celular\\
        
\item Explique lo que entiende por: transmisiones en baseband y en broadband e indique 4 diferencias
entre las dos tecnicas de transmisión\\

\item Indique 2 ejemplos de redes que transmiten en baseband y 2 que transmiten en broadband\\

\item Defina o explique: topología física y topología logica. \\

\item De un ejemplo de red que corresponda a una topología física bus y que tenga una topología
logica anillo.\\
    topología anillo : sala 

\item De un ejemplo de red que corresponda a una topolog ´ ´ıa f´ısica estrella y que tenga una topolog´ıa
logica bus.

\item Señale las ventajas y desventaja de los tipos de topolog ˜ ´ıas f´ısica de bus lineal, estrella, arbol, ´
malla y anillo.

\item De un ejemplo de una red que corresponda a una topolog ´ ´ıa f´ısica: bus, estrella, arbol, malla y ´
anillo.\\

\item Indique que significan las siglas BAN, WBAN, PAN, LAN, WLAN, MAN, WAN y GAN.\\
    BAN: Body Area Network\\
    WBAN: Wireless Body Area Network\\
    PAN: Personal Area Network\\
    LAN: Local Area Network\\
    WLAN: Wireless Local Area Network\\
    MAN: Metropolitan Area Network\\
    WAN: Wide Area Network\\
    GAN: Global Area Network\\


\item ¿Cuales son las funciones de un DTE y las de un DCE. Dé 2 ejemplos de cada uno?.\\

\item ¿A que tipos de redes (PAN, LAN, WLAN, etc.) se refieren los estándares IEEE 802.3, 802.5,
802.11, 802.15, 802.16?\\
    IEEE 802.3: MAN\\
    IEEE 802.5: LAN\\
    IEEE 802.11: WLAN\\
    IEEE 802.15: WPAN\\
    IEEE 802.16: WMAN\\

\item Indique cual designación 'Ethernet' (Fast, Giga, etc.) corresponden las siguientes extensiones
del estandar IEEE 802.3: 802.3u; 802.3ab; 802.3ae; 802.3ba.\\
    IEEE 802.3: Ethernet\\
    IEEE   802.3u: Fast Ethernet\\
    IEEE  802.3ab: Gigabit Ethernet\\
    IEEE 802.3ae: 10 Gigabit Ethernet\\
    IEEE 802.3ba: 100 Gigabit Ethernet\\
    
\item Indique la velocidad, tipo de cable, topología y distancia maxima del segmento para las 
siguientes especificaciones de capa física Ethernet: 10BASE2, 10BASE5, 100BASE-T, 100Base-TX,
100BASE-FX, 1000Base-T, 1000BASE-CX, 1000Base-LX, 10GBASE-S y 10GBASE-L.\\

\item ¿Por que la distancia máxima de segmento de red es mayor para 100BASE-FX que para 100BASE- ´
TX?.\\

\item Dibuje el formato del frame Ethernet indicando el tamano (en bytes) de cada uno de sus campos
y las funciones que cumplen.\\

\item ¿Cual es el mínimo y maximo tamaño de un frame Ethernet?.\\
    El mínimo son 84 bytes y el máximo 1542 bytes
\item ¿Por que es importante la etapa de preámbulo al comienzo de cada frame?.\\

\item Explique las diferencias entre direcciones logicas y direcciones físicas.

\item En relacion a una dirección MAC Ethernet, explique el propósito de los primeros 24 bits. ¿Qué
es OUI en una direccion MAC?, ¿qué indica una dirección MAC de destino FF-FF-FF-FF-FF- ´
FF?.\\

\end{enumerate}
\end{document}
