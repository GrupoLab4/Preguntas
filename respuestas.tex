\documentclass{udparticle}
\setlogo{EIT}
\headertext{Respuestas tarea de preparación para la Solemne}
\title{Tarea de preparación para la Solemne}
\author{Ignacio Yanjari, Dagoberto Navarrete, Ignacio López, Thomas Muñoz.}
\begin{document}
\maketitle
\begin{enumerate}
%aca se iran poniendo las respuestas, para esto escribir \item y su respuesta, eso si estarán en orden.
1.-Fuente:Emite el mensaje
Transductor de entrada: Convierte el mensaje en señales eléctricas
Transmisor: Pasa la señal al canal, mediante el proceso de modulación
Canal de Transmisión: Nexo eléctrico entre el emisor y receptor
Transductor de salida: transforma las señales para que puedan llegar a destino
Receptor: Realiza el proceso de demodulación 
2.-estructura en capas de hardware y software que facilita el intercambio de información entre sistemas
yo digo que en la 2 va esto : los protocolos describen los requerimientos e iteracciones precisas para que los
dispositivos se puedan comunicar en forma exitosa
7.- el estándar de jure es aprobado por una organización, mientras que el de facto es un estándar aceptado pero no legitimado por una organización
8.-Interfaz: es una conexión entre dos sistemas dando una comunicación en distintos niveles
Protocolo: Conjunto de reglas y estándares que controlan la comunicación entre sistemas
11.-ANSI: Instituto nacional estadounidense de estándares
TIA: Asociación de la industria de telecomunicaciones
EIA: Alianza de industrias electrónicas
ITU-T: Sector de normalización de las comunicaciones de la UIT
CCITT: Comité Consultivo Internacional Telegráfico y Telefónico
ISO: Organización Internacional de Normalización
IEEE: Instituto de Ingeniería Eléctrica y Electrónica
21.-BAN: Body Area Network
WBAN: Wireless Body Area Network
PAN: Personal Area Network
LAN: Local Area Network
WLAN: Wireless Local Area Network
MAN: Metropolitan Area Network
WAN: Wide Area Network
GAN: Global Area Network
23.- IEEE 802.3: MAN
IEEE 802.5: LAN
IEEE 802.11: WLAN
IEEE 802.15: WPAN
IEEE 802.16: WMAN
24.- IEEE 802.3: Ethernet
IEEE   802.3u: Fast Ethernet
IEEE  802.3ab: Gigabit Ethernet
IEEE 802.3ae: 10 Gigabit Ethernet
IEEE 802.3ba: 100 Gigabit Ethernet
28.- El mínimo son 84 bytes y el máximo 1542 bytes

\end{enumerate}
\end{document}
