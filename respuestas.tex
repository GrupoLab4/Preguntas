\documentclass{udparticle}
\setlogo{EIT}
\headertext{Respuestas tarea de preparación para la Solemne}
\title{Tarea de preparación para la Solemne}
\author{Ignacio Yanjari, Dagoberto Navarrete, Ignacio López, Thomas Muñoz.}
\usepackage{graphicx}
\usepackage{float}
\graphicspath{ {img/} }
\begin{document}
\maketitle
\begin{enumerate}
%aca se iran poniendo las respuestas, para esto escribir \item y su respuesta, eso si estarán en orden.

\item Describa cada uno de los componentes basicos de un sistema de comunicaciones \\
    Fuente:Emite el mensaje\\
    Transductor de entrada: Convierte el mensaje en señales eléctricas\\
    Transmisor: Pasa la señal al canal, mediante el proceso de modulación\\
    Canal de Transmisión: Nexo eléctrico entre el emisor y receptor\\
    Transductor de salida: transforma las señales para que puedan llegar a destino\\
    Receptor: Realiza el proceso de demodulación \\

\item ¿Que es una arquitectura de protocolos? \\
    Estructura en capas de hardware y software que facilita el intercambio de información para que los
    dispositivos se puedan comunicar en forma exitosa\\

\item Resuma las funciones de las siete capas del modelo OSI.\\
    -Aplicación:Conectividad entre red y humano\\
    -Presentación: Representacion común entre los datos transferidos y la capa de aplicacion\\
    -Sesión: Comunicacion entre dispositivos de la red\\
    -Transporte: Conexión extremo-a-extremo y fiabilidad de los datos\\
    -Red:Intercambia los datos individuales en la red de cada dispositivo final, determinacion de ruta e IP\\
    -Enlace de red: Describe los metodos para intercambiar datos, direccionamiento físico MAC y LLC\\
    -Física: Medios mecánico para la transmision y recepción\\
\item Resuma las funciones del modelo TCP/IP híbrido, incluyendo las 2 sub-capas de enlace de datos
(LLC y MAC).\\
	La sub-capa LLC(Software) toma los datos del protocolo de la red que generalmente son un paquete IPv4, y agrega 
	información de control para ayudar a entregar el paquete al nodo de destino. \\
	La sub-capa MAC(Hardware) le agrega una caracteristica más a c/computador el cual es la MAC

\item Mencione algunas ventajas y desventajas del modelo OSI de ISO\\
    -Ventajas:\\
        -Facilita un problema complejo ya que divide este en partes más simples y ordenadas\\
        -Evita los problemas de incompatibilidad\\
        -Los cambios de una capa no afectan las demás capas y éstas pueden evolucionar más rápido\\
    -Desventajas:\\
        -Las capas contienen demasiadas actividades redundantes, por ejemplo, el control de errores se integra en \\
         casi todas las capas siendo que tener un único control en la capa de aplicación o presentación sería suficiente\\
        -La gran cantidad de código que fue necesario para implantar el modelo OSI y su consecuente lentitud hizo que \\
         la palabra OSI fuera interpretada como "calidad pobre" contrastado con el modelo TCP/IP que fue mucho mas efectivo.\\
        -Tecnología desactualizada\\
        
\item Mencione algunas ventajas y desventajas del modelo de referencia TCP/IP híbrido.\\
    Ventajas:-Divide la capa de enlace de el modelo OSI en tipo WAN y LAN, a la misma vez LAN se subdivide en LLC y MAC
             -Contiene LLC y MAC el cual agrega una confiabilidad especial a el modelo ya que apoya el intercambio de datos
              entre equipos gracias a la direccion física
    Desventajas: -Contiene menos capas, por lo contiene una descripcion mas amplia de una red 
\item Indique la diferencia entre un estándar de  facto y un estándar de  jure\\
    El estándar de jure es aprobado por una organización, mientras que el de facto es un estándar aceptado pero 
    no legitimado por una organización\\
\item Defina o explique los conceptos de: interfaz y protocolo\\
    Interfaz: es una conexión entre dos sistemas dando una comunicación en distintos niveles\\
    Protocolo: Conjunto de reglas y estándares que controlan la comunicación entre sistemas\\
\item Explique en que consiste el proceso de encapsulamiento del mensaje\\
    El encapsulamiento rodea los datos con la información de protocolo necesaria antes de que se una al tránsito de la red.
    Por lo tanto, a medida que los datos se desplazan a través de las capas del modelo reciben encabezados, información final y otros
    tipos de información.\\
    Una vez que se envían los datos desde el origen, viajan a través de la capa de aplicación y recorren todas las demás capas en 
    sentido descendente. El empaquetamiento y el flujo de los datos que se intercambian experimentan cambios a medida que las capas 
    realizan sus funciones para los usuarios finales.\\

\itemIndique En que capas del modelo OSI las unidades de datos de los protocolos se llaman segmento,
frame, paquete y bits.\\
    REVISEN ESTO NO ESTOY MUY SEGURO D:
  Física -> Bits\\
  Enlace de red -> Paquete\\
  Red -> Frame\\
  Transporte -> Segmento\\

\item Relacionado con organismos de estandarizacion en redes LAN, ¿qu ´ e significan las siglas ANSI, ´
TIA, EIA, ITU-T, CCITT, FCC, ISO e IEEE?\\

    ANSI: Instituto nacional estadounidense de estándares\\
    TIA: Asociación de la industria de telecomunicaciones\\
    EIA: Alianza de industrias electrónicas\\
    ITU-T: Sector de normalización de las comunicaciones de la UIT\\
    CCITT: Comité Consultivo Internacional Telegráfico y Telefónico\\
    ISO: Organización Internacional de Normalización\\
    IEEE: Instituto de Ingeniería Eléctrica y Electrónica\\

\item Explique lo que entiende por: broadcast (difusión),multicast y unicast.\\
    -broadcast: se envia un paquete a todos los hosts conectados a la red\\
    -multicast: se envia un paquete a un grupo seleccionado de hosts conectados a la red\\
    -unicast : envia un paquete a un host determinado de la red\\
\item Defina y compare los medios simplex, half-duplex y full-duplex. De ejemplos de sistemas para ´
cada uno de estos casos.\\
    simplex: transmite solo en una direccion \\
        emjemplo: cuando se imprime a traves de un computador a una impresora\\
    half-duplex : Transmite en los 2 sentidos pero no a la vez \\
        ejemplo : wokitokis\\
    full-duplex : Transmite en los 2 sentidos simultaneamente\\
        ejemplo : Llamada en telefono celular\\
        
\item Explique lo que entiende por: transmisiones en baseband y en broadband e indique 4 diferencias
entre las dos tecnicas de transmisión\\
    -baseband:Son las señales que no sufren ningun proceso de modulacion a la salida de fuente que originan,
              es decir que son las señales transmitidas en su frecuencia original.
    -broandband:se refiere a la transmisión de datos en el cual se envían simultáneamente varias piezas de información, 
              con el objeto de incrementar la velocidad de transmisión efectiva,lleva más de una señal y cada una de 
              ellas se transmite en diferentes canales, hasta su número máximo de canal\\
    -Diferencias :
        -baseband transmite solo de 1 señal por un medio, pero broandband puede transmitir mas de 1 señal por el mismo medio\\
        -baseband funciona solo con sistema de 0 y 1 en cambio el broandband trabaja en forma de frecuencia,amplitud,etc\\
        -baseband se modela como 0 si estaba off y 1 si esta on en cambio el broandband se modela como una onda\\
        -baseband transmite direccionalmente en cambio broadband transmite bidireccionalmente\\

\item Indique 2 ejemplos de redes que transmiten en baseband y 2 que transmiten en broadband\\
    -baseband:\\
        - microfono conectado a parlante\\
        - Manipulador telegrafico\\
    -broadband:\\
        -Streaming\\
        -Envio archivos\\

\item Defina o explique: topología física y topología logica. \\
    -Topología físico 
        Se refiere a como esta contruida una red con tuberias,cables,ubicacion de pcs.
    -Topología logica
        Se enfoca en como estan conectados los pcs, configuracion de estos como por ejemplo las IP's,
        submascaras de red, puertos designados.
\item De un ejemplo de red que corresponda a una topología física bus y que tenga una topología
logica anillo.\\


\item De un ejemplo de red que corresponda a una topología física estrella y que tenga una topología
logica bus.

\item Señale las ventajas y desventaja de los tipos de topologías física de bus lineal, estrella, arbol, ´
malla y anillo.\\
    Tipo Bus:\\
        -Ventajas: -Ocupa menos cableado, es facil de arreglar \\
        -Desventajas: -Si un usuario desconecta su computadora o hay alguna falla como la rotura de un cable
                       la red deja de funcionar\\
                      -La velocidad de la red es muy baja\\
                      -las computadoras de la red no regeneran la señal sino que es generada por el cable
                       y los terminales de la topologia\
    Tipo Estrella:\\
        -Ventajas: -Si una computadora o cable falla no afecta a la red completa,solamente afecta a 
                    este equipo\\
                   -Facil deteccion de errores y reparacion\\
                   -Distribuye mejor el espacio ya que se puede ubicar el switch o HUB en el centro de
                    la red\\
            
        -Desventajas: - El numero de computadores conectados a la red se limita al hecho de cuantos puertos
                        tenga el switch o HUB de la red.\\
                      -Si el HUB o switch deja de funcionar toda la red quedará sin conexión.\\
                      -No es tan económica ya que es necesario mas cable para hacer las conexiones\\
    Tipo Arbol:\\
        -Ventajas:-Gracias a sus subdivisiones se puede limitar el acceso a cierta información
                   de algunos usuarios\\
                  -Es considerada a nivel estructural como la mejor de las redes ramificadas ya
                   que permite crear un orden jerárquico\\
        -Desventajas: - Si la columna principal llega a tener una falla se sufrirá una interrupcion
                        general a forma de red\\
                      - Llega a tener esta red un mantenimiento complicado con un alto costo ya que 
                        tiene un tamaño considerable\\
    Tipo malla:\\
        -Ventajas:  -Tiene redundancia permitiendo asi una mayor confiabilidad en la red\\
                    -No requiere de un computador presenta una falla respecto a cable no afecta a 
                     la red\\
                    -No existe ninguna interrupción entre las comunicaciones\\
        -Desventajas:-El costo de la red puede aumentar debido al cableo necesario para implementarla\\
        
    Tipo anillo:\\
        -Ventajas:-Los datos fluyen en una unica direccion\\
                  -Cada estacion recibe los datos y los retransmite al siguiente equipo\\
                  -Minimo embotellamiento de los datos en la red\\
        -Desventajas:-Como se encuentran unidos si falla un canal entre 2 nodos falla toda la red\\

\item De un ejemplo de una red que corresponda a una topología física: bus, estrella, arbol, malla y ´
anillo.\\

\item Indique que significan las siglas BAN, WBAN, PAN, LAN, WLAN, MAN, WAN y GAN.\\
    BAN: Body Area Network\\
    WBAN: Wireless Body Area Network\\
    PAN: Personal Area Network\\
    LAN: Local Area Network\\
    WLAN: Wireless Local Area Network\\
    MAN: Metropolitan Area Network\\
    WAN: Wide Area Network\\
    GAN: Global Area Network\\


\item ¿Cuales son las funciones de un DTE y las de un DCE. Dé 2 ejemplos de cada uno?.\\
    DCE:\\
        Puede comprender convertidores de señales, generadores de temporización,regeneradores de impulsos y dispositivos de control, 
        junto con equipos con otras funciones como protección contra errores o llamada y respuesta automática.\\
        Ejemplos:PC,impresora\\
    DTE:\\
        Es un Equipo Terminal de Datos.Es aquel componentedel circuito de datos que hace de fuente o destino de la información,
        este los procesa y los envia modificandolos.\\
        Ejemplos:switch,router\\
        
\item ¿A que tipos de redes (PAN, LAN, WLAN, etc.) se refieren los estándares IEEE 802.3, 802.5,
802.11, 802.15, 802.16?\\
    IEEE 802.3: MAN\\
    IEEE 802.5: LAN\\
    IEEE 802.11: WLAN\\
    IEEE 802.15: WPAN\\
    IEEE 802.16: WMAN\\

\item Indique cual designación 'Ethernet' (Fast, Giga, etc.) corresponden las siguientes extensiones
del estandar IEEE 802.3: 802.3u; 802.3ab; 802.3ae; 802.3ba.\\
    IEEE 802.3: Ethernet\\
    IEEE   802.3u: Fast Ethernet\\
    IEEE  802.3ab: Gigabit Ethernet\\
    IEEE 802.3ae: 10 Gigabit Ethernet\\
    IEEE 802.3ba: 100 Gigabit Ethernet\\
    
\item Indique la velocidad, tipo de cable, topología y distancia maxima del segmento para las 
siguientes especificaciones de capa física Ethernet: 10BASE2, 10BASE5, 100BASE-T, 100Base-TX,
100BASE-FX, 1000Base-T, 1000BASE-CX, 1000Base-LX, 10GBASE-S y 10GBASE-L.\\
    10BASE2\\ 
        -Tipo Cable: coaxial\\
        -Velocidad : 10 Mbit/s\\
        -Distancia máxima : 185 metros\\
        -Topología: recubierto con diélectrico (tipo bus)\\
    10Base5\\
        -Tipo cable: Coaxial\\
        -Velocidad: 10 Mbit/s\\
        -Distancia Máxima: 500 metros\\
        -Topología: recubierto con diélectrico (tipo bus)\\
    100BASE-T\\
        -Tipo cable: UTP\\
        -Velocidad: 100 Mbit/s\\
        -Distancia Máxima: 100 metros\\
        -Topología:Estrella\\
    100Base-TX\\
        -Tipo cable: UTP\\
        -Velocidad:100 Mbit/s\\
        -Distancia Máxima:100 metros\\
        -Topología:Estrella\\
    100BASE-FX\\
        -Tipo cable:Fibra óptica\\
        -Velocidad: 100 Mbit/s\\
        -Distancia Máxima: 2000 metros\\
        -Topología:No permite uso de hubs\\
    1000Base-T\\
        -Tipo cable: UTP\\
        -Velocidad: 1000Mb/s\\
        -Distancia Máxima:100 metros\\
        -Topología: Estrella\\
    1000BASE-CX\\
        -Tipo cable:STP\\
        -Velocidad: 1000 Mb/s\\
        -Distancia Máxima:25 metros\\
        -Topología:Estrella\\ 
    1000Base-LX\\
        -Tipo cable:fibra óptica\\
        -Velocidad: 1000 Mb/s\\
        -Distancia Máxima: 10 kilometros\\
        -Topología:No permite uso de HUB\\
    10GBASE-S\\
        -Tipo cable:Fibra óptica\\
        -Velocidad:10 Gbit/s\\
        -Distancia Máxima:300 metros\\
        -Topología:no permite uso de HUB\\
    10GBASE-L.\\
        -Tipo cable:Fibra óptica\\
        -Velocidad:  10 Gbit/s\\
        -Distancia Máxima: 40 kilometros\\
        -Topología:no permite uso de HUB\\

\item ¿Por que la distancia máxima de segmento de red es mayor para 100BASE-FX que para 100BASE- ´
TX?.\\
  La distancia de 100BASE-FX es mayor ya que en su composicion tiene 2 cables de fibra óptica multimodo para la recepcion y
  transmisión.En cambio el 100BASE-TX funciona gracias a cables de categoria cat6 y modelamiento de cable cruzado.

\item Dibuje el formato del frame Ethernet indicando el tamano (en bytes) de cada uno de sus campos
y las funciones que cumplen.\\
\begin{figure}[H]
	\centering
	\includegraphics[width=\textwidth]{frame.jpg}
	\caption{Frame}
\end{figure}\\
\item ¿Cual es el mínimo y maximo tamaño de un frame Ethernet?.\\
    El mínimo son 84 bytes y el máximo 1542 bytes
\item ¿Por que es importante la etapa de preámbulo al comienzo de cada frame?.\\
    Por que le indica al receptor el comienzo de un frame, para que este se prepare para recibirlo.
\item Explique las diferencias entre direcciones logicas y direcciones físicas.\\
    La direccion física MAC es un identificador de 48 bits que sirve para identificar de forma única a la tarjeta de red y no depende 
    del protocolo ni de la conexion utilizada en la red, en cambio la direccion logica(IP) si depende de estos.
    Tambien la (IP) no existe fisicamente en cambio la direccion fisica si se encuentra dentro de el pc
    La IP puede ser  variable en cambio la MAC no.-
    
    -La dirección logica(IP) no debe confundirse con la dirección MAC, porque la MAC es un identificador de 48 bits para identificar 
    de forma única  a la tarjeta de red y no depende del protocolo de conexión utilizado ni de la red.\\
    -La direecion fisica(MAC) es permanente y no varía, en cambio una direccion IP puede ser variable dependiendo del protocolo
    de conexión utlizado en la red.\\

    \item En relacion a una dirección MAC Ethernet, explique el propósito de los primeros 24 bits. ¿Qué es OUI en una direccion MAC?,
    ¿qué indica una dirección MAC de destino FF-FF-FF-FF-FF-FF?.\\
    Los primeros 24 bits son el denominado OUI (idenfitificador único de organizacion) el cual es comprado a la autoridad de el 
    el registro del instituto de Ingeniería Électrica y Electrónica(IEEE).Estos bits son un  identificador único,que identifica a 
    cada empresa u organización (llamados asignados) a nivel mundial y reserva un bloque en cada posible identificador derivado para 
    el uso exclusivo del asignado\\
    La  MAC FF-FF-FF-FF-FF-FF es la direccion asignada para transmitir de forma broadcast los paquetes dentro de una red.\\
    \item Calcule la máxima cantidad de frames que puede transmitir un nodo Ethernet en un canal de 10
    Mbps si en el campo de datos se tiene 46 bytes, no hay colisiones y el “gap” entre frames es de
    9.6 ms (12 bytes).\\
    Nos dicen que el canal es de 10 Mbps, si lo pasamos a bits serian 10x10^6 b. También nos dan el tamaño del campo de datos del
    frame de Ethernet el cual es de 46 B, por lo que el tamaño de todo el frame es de 84 B (contando los bytes del gap), los que
    transformados a bits serían 84x8 = 672 b.\\
    Entonces para calcular la máxima cantidad de frames que se pueden transmitir simplemente hay que dividir 10x10^6/672 = 14.880 b.
    \item Una red de comunicaciones que opera a 10 Mbps sólo puede cursar una media de 10.000 frames por minuto. Sabiendo que el
    tamãno medio de cada frame es de 10.000 bits, calcule el rendimiento de esta red.\\
    Primero, transformamos los 10 Mbps a 10x10^6 b, nos dicen que puede cursar una media de 10.000 frames por minuto, lo que es igual
    a 167 frames por segundo aproximadamente, y además no dan el tamaño medio de cada frame, que es de 10.000 b.\\
    Con todo esto si dividimos 10x10^6/10.000 nos dará la cantidad de frames que puede cursar pensando que el rendimiento fuera del
    100 por ciento, entonces con regla de tres calculamos que el porcentaje de rendimiento que tenemos es de 16,7.
    \item Explique en qué consiste el protocolo ARP. Además de algún ejemplo de un ataque a la seguri-
    dad de dicho protocolo.\\
    El protocolo ARP es un estándar TCP/IP que asocia las direcciones IP y MAC de los equipos conectados a la red. Un ejemplo de
    ataque utilizando ARP es el ARP-Spoofing consiste en enviar paquetes ARP falsos para que un router llene su tabla ARP con
    información falsa, de esta manera todos los datos que se supone llegarían a nuestra víctima llegaran al pc con la dirección MAC
    que hayamos especificado en el paquete falso.\\
    \item Explique por qué las entradas de una caché de ARP de un host de red se eliminan si no se
    utilizan durante un período de tiempo.\\
    Para dar espacio a nuevas direcciones MAC e IP de otros equipos que se estén usando más, ya que al estar en la tabla ARP el
    dispositivo no tendrá que mandar un paquete en busca de donde está el equipo de destino y el proceso de transferencia de datos
    será mas rápido.\\
    \item Utilizando el software Wireshark se pide realizar la captura de una secuencia ARP. Identifique
    cada una de las etapas del proceso.\\
    
    \item ¿Por qué debería existir menos colisiones en una red Ethernet conmutada comparada con una red Ethernet tradicional?.\\
    Por que en las redes Ethernet conmutadas los equipos no comparten un canal único, en cambio las redes tradicionales se comparte el
    canal de transmisión \\
    \item Compare las tasas de transmision de datos para la red Ethernet tradicional, Fast Ethernet y Gigabit Ethernet\\
    La tasa de transmision de la Ethernet normal es de 10Mbps, la de Fast Ethernet es de 100Mbps, y la Gigabit Ethernet es de 1Gbps\\
    \item Indique la configuracion de pines en conectores RJ45 para cables UTP cruzados FastEthernet y GigabitEthernet(falta)\\
    \item  Explique lo que entiende por la unidad decibel (dB)\\
    Es una relacion entre dos potencias electricas\\
    \item Si una senal viaja desde un punto A a un punto B. En el punto A la senal tiene una potencia de 100 mW y en el punto B tiene 90 mW. Indique el factor de atenuacion en dB(falta)\\
    \item Si la atenuacion de una senal a trav´es de un medio guiado es de -10 dB. Indique la potencia de la senal recibida si se transmiti´o a una potencia de 300 mW(falta)\\
    \item Explique lo que entiende por relacion senal a ruido SNR en dB\\
    Es la relacion entre la potencia que transmite y la potencia del ruido que la daña\\
    \item  Explique con sus propias palabras el Teorema de la capacidad de Shannon. \\
    El teorema de Shannon basicamente lo que nos indica es la maxima cantidad de datos que pueden ser enviados sin error y de calcula de la manera:\\
            C=BLog 2 (1 + S/N)\\
            C es la cantidad de datos enviados sin error\\
            B es el ancho de banda medidos en hertz\\
            S es la potencia de la señal util\\
            N es la potencia del ruido\\
    \item  Las condiciones de dise˜no de un sistema de comunicaciones que cuenta con un ancho de banda de 1MHz y presenta una SNR de 24dB es que alcance los 4Mbps.Indicar si este proyecto bajo las condiciones de ruido y ancho de banda es viable.(COMPLETAR) \\
    \item Un sistema de comunicaciones que posee un ancho de banda de 70 MHz y opera a 100 Mbps. Indicar la m´ınima SNR para que el sistema pueda funcionar(COMPLETAR) \\
    \item  Compare la maxima capacidad del canal del par trenzado y el cable coaxial.Para el par trenzado el ancho de banda es de 100MHz y la SNR es de 20dB.Para el cable coaxial el ancho de banda es de 250 MHz y su SNR es de 22 dB. (COMPLETAR)\\
    \item  ¿Cual es la capacidad para un canal de un teletipo de 300Hz de ancho de banda con una relacion senal a ruido de 3dB? (COMPLETAR)\\
    \item Sea un canal con una capacidad de 20Mbps.El ancho de banda de dicho canal es 3MHz¿Cual es la relacion senal a ruido admisible para conseguir la mencionada capacidad? (COMPLETAR)\\
    \item  Un sistema de comunicaciones trabaja a 100 Mbps y utiliza un ancho de banda de 70 MHz ¿Cual es la cota inferior de la relacion SNR para que el sistema pueda funcionar?. Calcule la eficiencia espectral maxima que se puede alcanzar con una SNR de 0 dB. (COMPLETAR)\\
    \item En que se diferencias los medios guiados de transmision de los no guiados\\
    Que en los medios guiados la transmision de infirmacion se logra a traves de cables interconectados entre los equipos, mientrs que en los no guiados se realiza de manera inalambrica ya sea mediante antenas, las cuales emiten los datos en forma de señal electromagnetica\\
    \item Indique 3 fuentes de distorsion de la señal en un cable UTP\\
    -Emisiones electromagneticas del entorno\\
    -Distancias mayores a 100 mt entre terminales\\
    -(COMPLETAR)\\
    \item Indique las diferencias entre los cables UTP, STP y FTP\\
    El cable UTP no pose ningun tipo de blindaje, el STP pose un cubierta exterior de aluminio que protege de señales electromagneticas exteriores, y el FTP pose blindaje tanto en los pares trenzados como en el exterior y asi evita cualquier tipo de distorcion que puedan sufrir los datos.\\
    \item  ¿Por que el cable coaxial es superior al par trenzado?.\\
    Es superior solo en el hecho que es mas inmune a señales electromagneticas del exterior ya que este posee materiales aislantes.\\
    \item Explique el concepto de impedacia de los medios de transmision.¿Porque es importante tener en cuenta este parametro al momento de interconectar equipos? \\
    La impedancia es la relacion entre la tension y intenciada de corriente, es importante tenerla en cuenta al momento de armar una red ya que si un medio recive una impedancia mayor a la maxima que admite, este se puede quemar y dañar toda la red.\\
    \item ¿Cual es la ventaja de un par trenzado blindado sobre el par trenzado normal?.\\
    El par trenzado blindado es resistente a las emisiones electromagneticas del exterior, por lo que sus datos no se veran afectados por estas emisiones.\\
    \item ¿A que se refieren las pruebas de NEXT, FEXT, PSNEXT y PSFEXT, ELFEXT, PS-ELFEXT en cable UTP?\\
    Son ruebas relacionadas con la atenuacion y el crosstalk, nos ayudan para determinar la longitud maxima que puede extenderse un cable.\\
  \item Señale las principales caracterısticas del cable de fibra optica monomodo.\\
  Posee un unico nucleo, el cual tiene un lazer led(COMPLEMENTAR)\\
  \item Señale las principales caracterısticas del cable de fibra optica multimodo. \\
  Posee multiples nucleos, los cuales estan formados por leds(  complementar)\\
  \item Senale las principales caracterısticas del cable de fibra optica multimodo conındice gradual(COMPLETAR)\\
  \item Explique lo que entiende por distorsion por atenuacion y distorsion por dispersion\\
  La distorcion por atenuacion se produce por una disminucion de la potencia de la señal de transmicion a lo largo del cable, mientras que la distorcion por dispercion se produce por un ensanchamiento de la onda, el cual aumenta con la longitud recorrida y con el ancho de la fuente.\\
  \item Señale los distintos tipos de problemas que podrıa sufrir una fibra optica\\
  -Perdidas debido a acoplamiento con dispoditivo emisor de luz
  -Perdidas debido a absorcion
  -Perdidas debido a una presion externa
  -Perdidas por radiacion debido a curvaturas 
  -Perdidas debido a no uniformidad en el nucleo
  -Perdidas por splicing
  -Perdidas por acoplamiento con el receptor
  \item ¿Que es y que importancia tiene Level 3 en nuestro paıs?\\
  Level 3 es un provedor mundial de servicios convergentes de datos, voz, video, data center, y es importante ya que es el mayor provedor de redes de sudamerica\\
  \item . De acuerdo al enlace http://submarinecablemap.com ¿Cuantos enlaces submarinos tiene chile?.
¿Cuales son sus nombres, que ciudades enlazan, quienes son los dueños y que longitud tienen
los cables?.\\
Posee 4 enlaces subamrinos:\\
-South America-1(SAm-1)\\
Dueños:Telefonica\\
Longitud: 25000 Km\\
Ciudades enlazadas: Arica y Valparaiso\\
-Pan American(PAN-AM)\\
Dueños: AT&T\\
Longitud: 7050 Km\\
Ciudades enlazadas: Arica\\
-South American Crossing (SAC)/Latin American Nautilus (LAN)\\
Dueños: Level 3\\
Longitud: 20000 Km\\
Ciudades enlazadas: Valparaiso\\
-South America Pacific Link (SAPL)\\
Dueños: Ocean Networks\\
Longitud: 17600 Km\\
Ciudades enlazadas: Valparaiso\\ 
\item Si la fibra óptica tiene una longitud de 500 metros, indicar cuál es la diferencia entre el camino 
óptico más corto y el camino  óptico más largo(No entender :c)\\
\item  Un haz de luz viaja a través de un medio menos denso. ¿Qué le ocurre a haz en cada uno de los siguientes casos? \\
a) El ángulo de incidencia es menor que el ángulo crıtico: Cuando el ángulo de incidencia es menor al crítico el haz de luz ingresa al medio más denso pero sufre refracción por lo que entra con un ángulo menor al que entro.\\
b) El ángulo de incidencia es igual al ángulo crıtico: El Haz de luz de luz va a ir paralelo a la separación entre los dos medios y no va a entrar al más denso.\\
c) El ángulo de incidencia es mayor que el ángulo crıtico: El haz de luz es reflejado por la superficie más densa por lo que no entra a esta.\\
\item  El índice de refracción del núcleo de una fibra óptica es de 1.5 y su revestimiento tiene un índice de 1.485. Indique: el máximo ángulo del cono de aceptación.(COMPLETAR)\\
\item  Indique las ventajas de la fibra óptica sobre el cable coaxial y el par trenzado.\\
-Grandes velocidades de transmisión de datos y mayor ancho de banda\\
-Inmune a interferencias electromagnéticas\\
-Baja perdida por atenuación\\
-Comunicación mas segura\\
-Cables mas ligeros\\
-Mayor periodo de vida en comparación con el coaxial o el par trenzado\\
\item  Indique 6 diferencias entre la Fibra Óptica monomodo y multimodo.\\
-La fibra monomodo posee un solo núcleo mientras que la monomodo posee múltiples núcleos.\\
-La fibra monomodo posee un mayor ancho de banda que la multimodo.\\
-La fibra monomodo posee menor atenuación que la multimodo.\\
-La fibra monomodo cubre mayores distancias que la multimodo.\\
-La fibra multimodo es de más fácil instalación.\\
-La fibra multimodo es de uso doméstico mientras que la monomodo es de uso industrial.\\
\item Indique los anchos de bandas disponibles para líneas dedicadas del tipo T y E.\\
La línea T es un tipo de línea que puede trasnportar mas datos que las lineas comunes, mientras que la e (NO HAY NADA :c)\\
\item  Indique las principales características del código Manchester.\\
Es un método de codificación eléctrica, esta codificación trabaja identificando los cambios de onda si pasa de positivo a negativo representa un uno, por el contrario si pasa de negativo a positivo representa un cero.\\
\item Bosqueje la codificación Manchester para el flujo de bits: 0001110101.(INSERTE IMG CON ESO)\\
\item  Bosqueje la codificación diferencial Manchester para el flujo de bits del problema anterior. Suponga que la línea se encuentra inicialmente en el estado bajo.(INSERT RESPUESTA)\\
\item Señale el estándar ANSI/EIA/TIA en vigencia actualmente relacionado con cableado estructurado.\\
El estándar actualmente utilizado es el ANSI/TIA/EIA-568-B tanto la variante B1 como la B2\\
\item En la implementación del cableado estructurado para un edificio de propósito general (no necesariamente comercial), ¿que estándares ANSI/EIA/TIA deben ser tomados en cuenta para los temas de cableado UTP, canalizaciones, salas de equipos, cuartos de telecomunicaciones y puestas a tierra? (indique los que corresponda, en forma separada).\\
- ANSI/TIA/EIA-568-B\\
- ANSI/TIA/EIA-569-A\\
- ANSI/TIA/EIA-570-A\\
- ANSI/TIA/EIA-606-A\\
- ANSI/TIA/EIA-607\\
- ANSI/TIA/EIA-758\\
\item Explique lo que entiende por banda ISM. Indique los rangos de frecuencia en que se encuentran las distintas bandas ISM definidas, además de sus aplicaciones típicas.\\
Son bandas de frecuencia para uso no comercial que operan en las frecuencias 902-928 MHz, 2,4-2,4835 GHz, 5,725-5,85 GHz, y por lo general se usan para teléfonos inalámbricos domesticos DECT, Bluetooth, y Wifi en redes locales.\\
\item  ¿Cual es la diferencia entre paridad par y paridad impar?.\\
En la paridad par se suman los 1 de la secuencia y si es impar se le agrega un 1 para que la suma sea par en caso de ser par se le agrega un 0, mientras que en la paridad impar si la suma de los 1 de la secuencia es par se le agrega un 1 para que la suma sea impar, en caso de ser impar se le agrega un 0 para que siga siendo impar\\
\item ¿Qué tipo de error no puede ser detectado por los métodos basados en el bit de paridad?.\\
No detecta cuando hay 2 errores en el grupo de bits ya que solo comprueba si es par o impar y al quitarle 2 bits queda igual que antes.\\
\item Una manera de detectar errores es transmitir los datos como un bloque de n filas de k bits por fila y agregar bits de paridad a cada fila y a cada columna. La esquina inferior derecha es un bit de paridad que verifica su fila y su columna. ¿Detectara este esquema todos los errores sencillos? ¿Los errores dobles?, ¿Los errores triples?\\
Este sistema detectaría los errores simples dobles y triples, pero no así los cuádruples ya que perdería un bloque cuadrado y no se daría cuenta de esto.\\
\item  ¿Qué tipo de error no puede ser detectado con una suma de comprobación?\\



  
  
    
    
\end{enumerate}
\end{document}
