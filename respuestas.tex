\documentclass{udparticle}
\setlogo{EIT}
\headertext{Respuestas tarea de preparación para la Solemne}
\title{Tarea de preparación para la Solemne}
\author{Ignacio Yanjari, Dagoberto Navarrete, Ignacio López, Thomas Muñoz.}
\begin{document}
\maketitle
\begin{enumerate}
%aca se iran poniendo las respuestas, para esto escribir \item y su respuesta, eso si estarán en orden.

\item Describa cada uno de los componentes basicos de un sistema de comunicaciones \\
    Fuente:Emite el mensaje
    Transductor de entrada: Convierte el mensaje en señales eléctricas
    Transmisor: Pasa la señal al canal, mediante el proceso de modulación
    Canal de Transmisión: Nexo eléctrico entre el emisor y receptor
    Transductor de salida: transforma las señales para que puedan llegar a destino
    Receptor: Realiza el proceso de demodulación 

\item ¿Que es una arquitectura de protocolos? \\
    estructura en capas de hardware y software que facilita el intercambio de información para que los
    dispositivos se puedan comunicar en forma exitosa

\item Resuma las funciones de las siete capas del modelo OSI de OSI.\\
    -Aplicación:Conectividad entre red y humano
    -Presentación: Representacion común entre los datos transferidos y la capa de aplicacion
    -Sesión: Organiza el dialogo y administra el intercambio de datos con la capa de presentación
    -Transporte: Transfiere,segmenta y reemsambla los datos para lo comunicacion entre los dispositivos finales
    -Red:Intercambia los datos individuales en la red de cada dispositivo final
    -Enlace de red: Describe los metodos para intercambiar datos
    -Física: Medios mecánico para la transmision y recepción
\item Resuma las funciones del modelo TCP/IP híbrido, incluyendo las 2 sub-capas de enlace de datos
(LLC y MAC).\\

\item Mencione algunas ventajas y desventajas del modelo OSI de ISO\\
    
\item Mencione algunas ventajas y desventajas del modelo de referencia TCP/IP híbrido.\\

\item Indique la diferencia entre un estándar de  facto y un estándar de  jure\\
    el estándar de jure es aprobado por una organización, mientras que el de facto es un estándar aceptado pero 
    no legitimado por una organización
\item Defina o explique los conceptos de: interfaz y protocolo\\
    Interfaz: es una conexión entre dos sistemas dando una comunicación en distintos niveles
    Protocolo: Conjunto de reglas y estándares que controlan la comunicación entre sistemas
\item
\item
\item
\item
\item
\item
\item
\item
\item
\item
\item
\item
\item
\item
\item
\item
\item
\item
\item
\item

11.-ANSI: Instituto nacional estadounidense de estándares
TIA: Asociación de la industria de telecomunicaciones
EIA: Alianza de industrias electrónicas
ITU-T: Sector de normalización de las comunicaciones de la UIT
CCITT: Comité Consultivo Internacional Telegráfico y Telefónico
ISO: Organización Internacional de Normalización
IEEE: Instituto de Ingeniería Eléctrica y Electrónica
21.-BAN: Body Area Network
WBAN: Wireless Body Area Network
PAN: Personal Area Network
LAN: Local Area Network
WLAN: Wireless Local Area Network
MAN: Metropolitan Area Network
WAN: Wide Area Network
GAN: Global Area Network
23.- IEEE 802.3: MAN
IEEE 802.5: LAN
IEEE 802.11: WLAN
IEEE 802.15: WPAN
IEEE 802.16: WMAN
24.- IEEE 802.3: Ethernet
IEEE   802.3u: Fast Ethernet
IEEE  802.3ab: Gigabit Ethernet
IEEE 802.3ae: 10 Gigabit Ethernet
IEEE 802.3ba: 100 Gigabit Ethernet
28.- El mínimo son 84 bytes y el máximo 1542 bytes

\end{enumerate}
\end{document}
